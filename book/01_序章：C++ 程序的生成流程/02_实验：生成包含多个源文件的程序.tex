% Licensed under the Creative Commons Attribution Share Alike 4.0 International.
% See the LICENSE file in the repository root for full license text.

\section{实验:生成包含多个源文件的程序}

包含多个源文件的 C++ 程序\emph{生成(build)}流程可以用图 \ref{fig:build-flow} 示意。该图将生成流程概括为以下三个步骤。

\begin{enumerate}
	\item \emph{预处理(preprocess)}。\emph{预处理器(preprocessor)}处理源文件中的 \lstinline[language={[17]C++}]{#include}、\lstinline[language={[17]C++}]{#define} 等预处理指令,对 \lstinline[language={[17]C++}]{#include} 指令、宏指令进行\textbf{文本替换}。
	\item \emph{编译(compile)}。\emph{编译器(compiler)}\textbf{独立地}处理各源文件,将每个源文件编译为中间文件。
	\item \emph{链接(link)}。\emph{链接器(linker)}处理\textbf{所有}需要的中间文件,最后生成可执行文件。
\end{enumerate}

\begin{figure}
	\centering
	\includegraphics[scale=0.15]{assets/build-flow}
	\caption{多个 C++ 源文件的生成流程示意图。}
	\label{fig:build-flow}
\end{figure}

一般认为,C++ 程序的生成流程\footnote{没有讨论单个源文件经过编译得到单个中间文件时,习惯上用“编译”替代“生成”。所以前文中的 \lstinline[language={}]{g++} 编译器实际上具有完整的生成功能,不止可以编译,还可以预处理、链接;前文中的“编译日志”也表示“生成日志”。}包括预处理、编译、汇编、链接四个步骤,此处我们将汇编这一步骤并入编译,以突出之后的实验要讨论的问题。

\subsection*{实验步骤}

\begin{enumerate}
	\item 新建项目。打开 Dev-C++,点击菜单栏中的“文件”、“新建”、“项目”,选择“Empty Project”,并为项目取一个好名称,如图 \ref{fig:multi-source-1} 所示。然后点击确定,通过“另存为”对话框选择项目保存的路径。

	\begin{figure}
		\centering
		\includegraphics[width=0.75\linewidth]{assets/multi-source-1}
		\caption{新建项目对话框。}
		\label{fig:multi-source-1}
	\end{figure}

	\item 新建源文件。在“项目管理”窗口中,右键点击项目,点击弹出菜单中的 “New File”。立即保存新建的文件到默认路径,并为它取一个好名称。重复该步骤,得到如图 \ref{fig:multi-source-2} 所示的结果。

	\begin{figure}
		\centering
		\includegraphics[width=0.2\linewidth]{assets/multi-source-2}
		\caption{新建源文件后的“项目管理”窗口。}
		\label{fig:multi-source-2}
	\end{figure}

	\item 编写代码。

	\begin{lstlisting}[language={[17]C++}, moreemph={[2]another}]
// main.cpp
#include <iostream>

#include "another.h"

int main()
{
	long long ago { another("figure emerged") };
	std::cout << ago << std::endl;
}
	\end{lstlisting}

	\begin{lstlisting}[language={[17]C++}, moreemph={[2]another}]
// another.h
long long another(const char*);
	\end{lstlisting}

	\begin{lstlisting}[language={[17]C++}, moreemph={[2]another, strlen}]
// another.cpp
#include <cstring>

long long another(const char* str)
{
	return std::strlen(str);
}
	\end{lstlisting}

	\item 编译并运行代码。点击菜单栏中的“运行”、“编译运行”,可以看到“编译日志”窗口弹出,调用的命令记录如下。

	\begin{lstlisting}[language={}]
生成 makefile...
--------
- 文件名: C:\Users\lyche\Desktop\GoodName\Makefile.win

正在处理makefile...
--------
- makefile处理器: C:\Program Files (x86)\Embarcadero\Dev-Cpp\TDM-GCC-64\bin\mingw32-make.exe
- 命令: mingw32-make.exe -j16 -f "C:\Users\lyche\Desktop\GoodName\Makefile.win" all

g++.exe -c another.cpp -o another.o -I"C:/Program Files (x86)/Embarcadero/Dev-Cpp/TDM-GCC-64/include"

g++.exe -c main.cpp -o main.o -I"C:/Program Files (x86)/Embarcadero/Dev-Cpp/TDM-GCC-64/include"

g++.exe another.o main.o -o GoodName.exe -L"C:/Program Files (x86)/Embarcadero/Dev-Cpp/TDM-GCC-64/lib" -static-libgcc
	\end{lstlisting}
\end{enumerate}

\subsection*{讨论}

\begin{enumerate}
	\item 各源文件的编译是独立进行的。编译日志的第 10 行只编译了 \lstinline[language={}]{another.cpp},而第 12 行只编译了 \lstinline[language={}]{main.cpp}。

	\item 编译各源文件时不涉及其他中间文件或其他库文件。编译日志的第 10 行和第 12 行均不包含 \lstinline[language={}]{-L} 参数,只有在链接时(第 14 行)才会通过 \lstinline[language={}]{-L} 参数给出需要额外链接的库所在的目录。

	\item \lstinline[language={}]{another.h} 的作用。

	我们只在 \lstinline[language={}]{main.cpp} 中通过 \lstinline[language={[17]C++}]{#include} 指令引用了 \lstinline[language={}]{another.h}。将这句 \lstinline[language={[17]C++}]{#include} 指令展开(即进行文本替换),可以得到如下等效的 \lstinline[language={}]{main.cpp} 文件。

	\begin{lstlisting}[language={[17]C++}, moreemph={[2]another}]
// main.cpp
#include <iostream>

long long another(const char*);

int main()
{
	long long ago { another("figure emerged") };
	std::cout << ago << std::endl;
}
	\end{lstlisting}

	第 4 行的 \lstinline[language={[17]C++}, moreemph={[2]another}]{long long another(const char*);} 称为函数的\emph{声明(declaration)},它提供了函数的\emph{原型(prototype)},即返回值类型、函数名、参数列表。因为各源文件的编译是独立进行的,所以可以预见,如果不\emph{声明(declare)}该函数,编译器在调用该函数相关的代码时就会因为不知道函数的返回值类型和参数列表而报错。这就是常说的“先声明,后使用”的原则。

	综上,\lstinline[language={}]{another.h} 的作用是为其他源文件提供 \lstinline[language={}]{another.cpp} 中的函数的声明。

	\item 分离声明与\emph{实现(implementation)}的原因。

	链接时,链接器检查到 \lstinline[language={}]{main.cpp} 调用了 \lstinline[language={[17]C++}, moreemph={[2]another}]{another} 函数,而我们是在 \lstinline[language={}]{another.cpp} 中具体\emph{实现(implement)} \lstinline[language={[17]C++}, moreemph={[2]another}]{another} 函数的,于是链接器将编译 \lstinline[language={}]{another.cpp} 得到的部分中间结果并入可执行文件中。

	为什么不只在 \lstinline[language={}]{another.h} 中实现 \lstinline[language={[17]C++}, moreemph={[2]another}]{another} 函数,而要把声明和实现分离在两个文件中?在本例中,这似乎是可行的,进行文本替换后唯一的源文件 \lstinline[language={}]{main.cpp} 将变为:

	\begin{lstlisting}[language={[17]C++}, moreemph={[2]another, strlen}]
// main.cpp
#include <iostream>
#include <cstring>

long long another(const char* str)
{
	return std::strlen(str);
}

int main()
{
	long long ago { another("figure emerged") };
	std::cout << ago << std::endl;
}
	\end{lstlisting}

	问题实际上就在于文本替换。如果存在另一个源文件 \lstinline[language={}]{yet_another.cpp},它也通过 \lstinline[language={[17]C++}]{#include} 指令包含了 \lstinline[language={}]{another.h},则文本替换后,\lstinline[language={[17]C++}, moreemph={[2]another}]{another} 函数将存在两个实现:这是链接器不允许的。

	综上,声明与实现分离的原因是避免展开 \lstinline[language={[17]C++}]{#include} 指令后在多个源文件中出现重复的函数实现\footnote{可以使用 \lstinline[language={[17]C++}]{inline} 关键字告诉编译器这个函数可以因为多次 \lstinline[language={[17]C++}]{#include} 重复定义。尽管如此,一般仍然会分离函数的声明和实现,以避免展开 \lstinline[language={[17]C++}]{#include} 指令产生大量实现代码,从而提升编译速度。相关知识属于 C++ 语法范畴,此处不予详细讨论。}。
\end{enumerate}
